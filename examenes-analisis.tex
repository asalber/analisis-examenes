% Options for packages loaded elsewhere
\PassOptionsToPackage{unicode}{hyperref}
\PassOptionsToPackage{hyphens}{url}
\PassOptionsToPackage{dvipsnames,svgnames,x11names}{xcolor}
%
\documentclass[
  a4paper,
]{scrreport}

\usepackage{amsmath,amssymb}
\usepackage{lmodern}
\usepackage{iftex}
\ifPDFTeX
  \usepackage[T1]{fontenc}
  \usepackage[utf8]{inputenc}
  \usepackage{textcomp} % provide euro and other symbols
\else % if luatex or xetex
  \usepackage{unicode-math}
  \defaultfontfeatures{Scale=MatchLowercase}
  \defaultfontfeatures[\rmfamily]{Ligatures=TeX,Scale=1}
\fi
% Use upquote if available, for straight quotes in verbatim environments
\IfFileExists{upquote.sty}{\usepackage{upquote}}{}
\IfFileExists{microtype.sty}{% use microtype if available
  \usepackage[]{microtype}
  \UseMicrotypeSet[protrusion]{basicmath} % disable protrusion for tt fonts
}{}
\makeatletter
\@ifundefined{KOMAClassName}{% if non-KOMA class
  \IfFileExists{parskip.sty}{%
    \usepackage{parskip}
  }{% else
    \setlength{\parindent}{0pt}
    \setlength{\parskip}{6pt plus 2pt minus 1pt}}
}{% if KOMA class
  \KOMAoptions{parskip=half}}
\makeatother
\usepackage{xcolor}
\setlength{\emergencystretch}{3em} % prevent overfull lines
\setcounter{secnumdepth}{5}
% Make \paragraph and \subparagraph free-standing
\ifx\paragraph\undefined\else
  \let\oldparagraph\paragraph
  \renewcommand{\paragraph}[1]{\oldparagraph{#1}\mbox{}}
\fi
\ifx\subparagraph\undefined\else
  \let\oldsubparagraph\subparagraph
  \renewcommand{\subparagraph}[1]{\oldsubparagraph{#1}\mbox{}}
\fi


\providecommand{\tightlist}{%
  \setlength{\itemsep}{0pt}\setlength{\parskip}{0pt}}\usepackage{longtable,booktabs,array}
\usepackage{calc} % for calculating minipage widths
% Correct order of tables after \paragraph or \subparagraph
\usepackage{etoolbox}
\makeatletter
\patchcmd\longtable{\par}{\if@noskipsec\mbox{}\fi\par}{}{}
\makeatother
% Allow footnotes in longtable head/foot
\IfFileExists{footnotehyper.sty}{\usepackage{footnotehyper}}{\usepackage{footnote}}
\makesavenoteenv{longtable}
\usepackage{graphicx}
\makeatletter
\def\maxwidth{\ifdim\Gin@nat@width>\linewidth\linewidth\else\Gin@nat@width\fi}
\def\maxheight{\ifdim\Gin@nat@height>\textheight\textheight\else\Gin@nat@height\fi}
\makeatother
% Scale images if necessary, so that they will not overflow the page
% margins by default, and it is still possible to overwrite the defaults
% using explicit options in \includegraphics[width, height, ...]{}
\setkeys{Gin}{width=\maxwidth,height=\maxheight,keepaspectratio}
% Set default figure placement to htbp
\makeatletter
\def\fps@figure{htbp}
\makeatother

\usepackage{venndiagram}
\newcommand{\NN}{\mathbb{N}}
\newcommand{\ZZ}{\mathbb{Z}}
\newcommand{\QQ}{\mathbb{Q}}
\newcommand{\RR}{\mathbb{R}}
\newcommand{\CC}{\mathbb{C}}
\DeclareMathOperator{\operatorname{Int}}{Int}
\DeclareMathOperator{\operatorname{Ext}}{Ext}
\DeclareMathOperator{\operatorname{Fr}}{Fr}
\DeclareMathOperator{\Adh}{Adh}
\DeclareMathOperator{\Ac}{Ac}
\DeclareMathOperator{\sen}{sen}
\makeatletter
\@ifpackageloaded{tcolorbox}{}{\usepackage[many]{tcolorbox}}
\@ifpackageloaded{fontawesome5}{}{\usepackage{fontawesome5}}
\definecolor{quarto-callout-color}{HTML}{909090}
\definecolor{quarto-callout-note-color}{HTML}{0758E5}
\definecolor{quarto-callout-important-color}{HTML}{CC1914}
\definecolor{quarto-callout-warning-color}{HTML}{EB9113}
\definecolor{quarto-callout-tip-color}{HTML}{00A047}
\definecolor{quarto-callout-caution-color}{HTML}{FC5300}
\definecolor{quarto-callout-color-frame}{HTML}{acacac}
\definecolor{quarto-callout-note-color-frame}{HTML}{4582ec}
\definecolor{quarto-callout-important-color-frame}{HTML}{d9534f}
\definecolor{quarto-callout-warning-color-frame}{HTML}{f0ad4e}
\definecolor{quarto-callout-tip-color-frame}{HTML}{02b875}
\definecolor{quarto-callout-caution-color-frame}{HTML}{fd7e14}
\makeatother
\makeatletter
\makeatother
\makeatletter
\@ifpackageloaded{bookmark}{}{\usepackage{bookmark}}
\makeatother
\makeatletter
\@ifpackageloaded{caption}{}{\usepackage{caption}}
\AtBeginDocument{%
\ifdefined\contentsname
  \renewcommand*\contentsname{Indice de contenidos}
\else
  \newcommand\contentsname{Indice de contenidos}
\fi
\ifdefined\listfigurename
  \renewcommand*\listfigurename{Listado de Figuras}
\else
  \newcommand\listfigurename{Listado de Figuras}
\fi
\ifdefined\listtablename
  \renewcommand*\listtablename{Listado de Tablas}
\else
  \newcommand\listtablename{Listado de Tablas}
\fi
\ifdefined\figurename
  \renewcommand*\figurename{Figura}
\else
  \newcommand\figurename{Figura}
\fi
\ifdefined\tablename
  \renewcommand*\tablename{Tabla}
\else
  \newcommand\tablename{Tabla}
\fi
}
\@ifpackageloaded{float}{}{\usepackage{float}}
\floatstyle{ruled}
\@ifundefined{c@chapter}{\newfloat{codelisting}{h}{lop}}{\newfloat{codelisting}{h}{lop}[chapter]}
\floatname{codelisting}{Listado}
\newcommand*\listoflistings{\listof{codelisting}{Listado de Listatdos}}
\usepackage{amsthm}
\theoremstyle{definition}
\newtheorem{exercise}{Ejercicio}[chapter]
\theoremstyle{remark}
\renewcommand*{\proofname}{Prueba}
\newtheorem*{remark}{Observación}
\newtheorem*{solution}{Solución}
\makeatother
\makeatletter
\@ifpackageloaded{caption}{}{\usepackage{caption}}
\@ifpackageloaded{subcaption}{}{\usepackage{subcaption}}
\makeatother
\makeatletter
\@ifpackageloaded{tcolorbox}{}{\usepackage[many]{tcolorbox}}
\makeatother
\makeatletter
\@ifundefined{shadecolor}{\definecolor{shadecolor}{rgb}{.97, .97, .97}}
\makeatother
\makeatletter
\makeatother
\ifLuaTeX
\usepackage[bidi=basic]{babel}
\else
\usepackage[bidi=default]{babel}
\fi
\babelprovide[main,import]{spanish}
% get rid of language-specific shorthands (see #6817):
\let\LanguageShortHands\languageshorthands
\def\languageshorthands#1{}
\ifLuaTeX
  \usepackage{selnolig}  % disable illegal ligatures
\fi
\IfFileExists{bookmark.sty}{\usepackage{bookmark}}{\usepackage{hyperref}}
\IfFileExists{xurl.sty}{\usepackage{xurl}}{} % add URL line breaks if available
\urlstyle{same} % disable monospaced font for URLs
\hypersetup{
  pdftitle={Exámenes de Análisis Matemático},
  pdfauthor={Alfredo Sánchez Alberca},
  pdflang={es},
  colorlinks=true,
  linkcolor={blue},
  filecolor={Maroon},
  citecolor={Blue},
  urlcolor={Blue},
  pdfcreator={LaTeX via pandoc}}

\title{Exámenes de Análisis Matemático}
\author{Alfredo Sánchez Alberca}
\date{1/11/2022}

\begin{document}
\begin{titlepage}

%\AddToShipoutPicture*{\put(0,0){\includegraphics[scale=0.8]{img/background2}}} % Imagen de fondo, requiere el paquete eso-pic.
\begin{center}
\vspace*{5cm}

\Huge
{\textbf{\textsf{Exámenes de Análisis Matemático}}}

\vspace{0.5cm}
\LARGE
{\textbf{\textsf{}}}

\vspace{1.5cm}

\includegraphics[width=0.4\textwidth]{img/logos/infinito.png}
\end{center}

\vfill

\begin{flushleft}
\begin{tabular}{ll}
\includegraphics[width=0.1\textwidth]{img/logos/aprendeconalf.png} & \parbox[b]{5cm}{\Large\textsf{Alfredo
Sánchez
Alberca}\\ \textsf{asalber@ceu.es} \\ \textsf{https://aprendeconalf.es}}
\end{tabular}
\end{flushleft}
\end{titlepage}\ifdefined\Shaded\renewenvironment{Shaded}{\begin{tcolorbox}[interior hidden, boxrule=0pt, breakable, enhanced, frame hidden, sharp corners, borderline west={3pt}{0pt}{shadecolor}]}{\end{tcolorbox}}\fi

\renewcommand*\contentsname{Indice de contenidos}
{
\hypersetup{linkcolor=}
\setcounter{tocdepth}{2}
\tableofcontents
}
\bookmarksetup{startatroot}

\hypertarget{prefacio}{%
\chapter*{Prefacio}\label{prefacio}}
\addcontentsline{toc}{chapter}{Prefacio}

Colección de exámenes de Análisis Matemático Real del grado en
Ingeniería Matemática.

\bookmarksetup{startatroot}

\hypertarget{examen-del-14-11-2022}{%
\chapter{Examen del 14-11-2022}\label{examen-del-14-11-2022}}

\leavevmode\vadjust pre{\hypertarget{exr-1}{}}%
\begin{exercise}[]\label{exr-1}

Calcular los puntos de acumulación del conjunto
\(A=[0,1]\cup \left\{\frac{n}{n-1}: n\in\mathbb{N}, n\geq 2\right\}\).
¿Es un conjunto cerrado? ¿Y abierto?

\end{exercise}

\begin{tcolorbox}[enhanced jigsaw, colbacktitle=quarto-callout-tip-color!10!white, breakable, toprule=.15mm, leftrule=.75mm, opacitybacktitle=0.6, left=2mm, colframe=quarto-callout-tip-color-frame, title=\textcolor{quarto-callout-tip-color}{\faLightbulb}\hspace{0.5em}{Solución}, colback=white, toptitle=1mm, bottomtitle=1mm, titlerule=0mm, coltitle=black, opacityback=0, arc=.35mm, rightrule=.15mm, bottomrule=.15mm]
Veamos primero que todos los puntos del intervalo \([0,1]\) son puntos
de acumulación. Sea \(x\in[0,1]\). Entonces para cualquier
\(\varepsilon>0\), por la densidad de los números reales, el entorno
reducido \((x-\varepsilon,x+\varepsilon)\setminus\{x\}\) contiene puntos
de \([0,1]\), y por tanto \(x\) es un punto de acumulación de \([0,1]\).

Veamos ahora que el conjunto
\(B=\left\{\frac{n}{n-1}: n\in\mathbb{N}, n\geq 2\right\} = \left\{1+\frac{1}{n-1}: n\in\mathbb{N}, n\geq 2\right\}\)
solo tiene \(1\) como punto de acumulación. En primer lugar, \(1\) es
punto de acumulación, ya que para cualquier \(\varepsilon>0\),
\((1-\varepsilon, 1+\varepsilon)\setminus \{1\}\) contiene puntos de
\(A\). Para verlo, basta aplicar la propiedad arquimediana, por la que
existe \(n\in\mathbb{N}\) tal que \(\frac{1}{n}<\varepsilon\), de manera
que \(1+\frac{1}{n}<1+\varepsilon\), y por tanto
\((1-\varepsilon, 1+\varepsilon)\setminus \{1\}\cap B\neq \emptyset\).

Si \(x<1\), tomando \(\varepsilon=|x-1|\) el entorno reducido
\((x-\varepsilon, x+\varepsilon)\setminus\{x\}\) no contiene puntos de
\(B\). Del mismo modo, si \(x>2\), tomando \(\varepsilon=|x-2|\) el
entorno reducido \((x-\varepsilon, x+\varepsilon)\setminus\{x\}\)
tampoco contiene puntos de \(B\). Finalmente, si \(1<x\leq 2\), por la
propiedad arquimediana, existe \(n\in\mathbb{N}\) tal que
\(\frac{1}{n}\leq x<\frac{1}{n-1}\). Tomando
\(\varepsilon=\min(\{|x-\frac{1}{n}|,|x-\frac{1}{n-1}|\})\) también se
tiene que el entorno reducido
\((x-\varepsilon, x+\varepsilon)\setminus\{x\}\) no contiene puntos de
\(B\). Por tanto, \(1\) es el único punto de acumulación de \(B\).

Así pues, \(\operatorname{Ac}(A)=[0,1]\), y como
\(\operatorname{Ac}(A)\subseteq A\), \(A\) es cerrado ya que contienen a
todos sus puntos de acumulación (ver
\href{https://aprendeconalf.es/analisis-manual/topologia-reales.html\#thm-conjunto-cerrado-puntos-acumulacion}{teorema}),
y por tanto, no puede ser abierto ya que los únicos conjuntos cerrados y
abiertos a la vez son \(\mathbb{R}\) y \(\emptyset\).
\end{tcolorbox}

\leavevmode\vadjust pre{\hypertarget{exr-2}{}}%
\begin{exercise}[]\label{exr-2}

Dada la sucesión \(\left(\frac{1}{2^n}\right)_{n=1}^\infty\),

\begin{enumerate}
\def\labelenumi{\alph{enumi}.}
\tightlist
\item
  Calcular, si existen, el supremo, ínfimo, máximo y mínimo del conjunto
  de sus términos.
\item
  Demostrar que la sucesión converge a \(0\).
\end{enumerate}

\end{exercise}

\begin{tcolorbox}[enhanced jigsaw, colbacktitle=quarto-callout-tip-color!10!white, breakable, toprule=.15mm, leftrule=.75mm, opacitybacktitle=0.6, left=2mm, colframe=quarto-callout-tip-color-frame, title=\textcolor{quarto-callout-tip-color}{\faLightbulb}\hspace{0.5em}{Solución}, colback=white, toptitle=1mm, bottomtitle=1mm, titlerule=0mm, coltitle=black, opacityback=0, arc=.35mm, rightrule=.15mm, bottomrule=.15mm]

\begin{enumerate}
\def\labelenumi{\alph{enumi}.}
\item
  La sucesión es monótona decreciente, ya que \(\forall n\in\mathbb{N}\)
  \(2^n<2^{n+1}\), y por tanto,
  \(x_n=\frac{1}{2^n}<\frac{1}{2^{n+1}}=x_{n+1}\). Así pues, el primer
  término de la sucesión \(x_1=1/2\) es su máximo, y por tanto el
  supremo.

  Veamos ahora que \(0\) es ínfimo por reducción al absurdo. En primer
  lugar, \(0\) es una cota inferior de la sucesión, pues todos sus
  términos son positivos. Supongamos ahora que existe otra cota inferior
  \(c\in\mathbb{R}\) tal que \(c>0\). Por la propiedad arquimediana,
  existe \(n\in\mathbb{N}\) tal que \(\frac{1}{n}< c\). Ahora bien, como
  \(n<2^n\) \(\forall n\in\mathbb{N}\), se tiene que
  \(\frac{1}{2^n}<\frac{1}{n}< c\), por lo que el termino \(n\) de la
  sucesión es menor que \(c\), lo que contradice que sea cota inferior.
  Así pues, \(0\) es el ínfimo. Sin embargo, la sucesión no tiene
  mínimo, pues \(x_n\neq 0\) \(\forall n\in\mathbb{N}\).
\item
  Como la sucesión es monótona decreciente y está acotada inferiormente,
  por el
  \href{https://aprendeconalf.es/analisis-manual/sucesiones.html\#thm-convergencia-monotona}{teorema
  de la convergencia de una sucesión monónota} la sucesión converge y
  \(\lim_{n\to\infty}x_n = \inf(\{x_n:n\in\mathbb{N}\})=0\).
\end{enumerate}

\end{tcolorbox}

\leavevmode\vadjust pre{\hypertarget{exr-3}{}}%
\begin{exercise}[]\label{exr-3}

La rentabilidad de un bono cada año, en porcentaje, viene dada por la
sucesión recurrente \(x_1=3\) y \(x_{n+1}=\sqrt{\frac{x_n}{2}+3}\).
¿Hacia dónde converge la rentabilidad del bono a medida que pasa el
tiempo?

\end{exercise}

\begin{tcolorbox}[enhanced jigsaw, colbacktitle=quarto-callout-tip-color!10!white, breakable, toprule=.15mm, leftrule=.75mm, opacitybacktitle=0.6, left=2mm, colframe=quarto-callout-tip-color-frame, title=\textcolor{quarto-callout-tip-color}{\faLightbulb}\hspace{0.5em}{Solución}, colback=white, toptitle=1mm, bottomtitle=1mm, titlerule=0mm, coltitle=black, opacityback=0, arc=.35mm, rightrule=.15mm, bottomrule=.15mm]
Veamos que primero que la sucesión es monótona decreciente.
\(x_1=3>x_n= \sqrt{\frac{3}{2}+3} = 2.12\). Supongamos ahora que
\(x_{n-1}>x_n\). Entonces

\begin{align*}
x_{n-1}>x_n &\Rightarrow \frac{x_{n-1}} {2}>\frac{x_n}{2} \Rightarrow \frac{x_{n-1}}{2}+3>\frac{x_n}{2}+3\\  & \Rightarrow \sqrt{\frac{x_{n-1}}{2}+3}>\sqrt{\frac{x_n}{2}+3} \Rightarrow x_n>x_{n+1}.
\end{align*} \[
\]

Por otro lado, es fácil ver que la sucesión está acotada inferiormente
por \(0\) pues todos los términos son positivos. Así pues, por el
\href{https://aprendeconalf.es/analisis-manual/sucesiones.html\#thm-convergencia-monotona}{teorema
de la convergencia de una sucesión monónota}, la sucesión converge a un
número \(x\in\mathbb{R}\). Para calcular el límite, aprovechando la
recurrencia de la sucesión se tiene

\[
x=\lim_{n\to\infty} x_n = \lim_{n\to\infty} x_{n+1} = \lim_{n\to\infty} \sqrt{\frac{x_n}{2}+3} = \sqrt{\lim_{n\to\infty}\frac{x_n}{2}+3} = \sqrt{\frac{x}{2}+3}
\]

Así pues, se cumple que \(x=\sqrt{\frac{x}{2}+3}\), y de ello se deduce

\[
x=\sqrt{\frac{x}{2}+3} \Rightarrow x^2=\frac{x}{2}+3 \Rightarrow x^2-3 = \frac{x}{2} \Rightarrow 2x^2-x-6 =0,\]

y resolviendo la ecuación se tiene \(x=-3/2\) y \(x=2\). Como todos los
términos de la sucesión son positivos, es imposible que converja a
\(-3/2\), y por tanto la rentabilidad del bono converge al \(2\)\%.
\end{tcolorbox}

\leavevmode\vadjust pre{\hypertarget{exr-4}{}}%
\begin{exercise}[]\label{exr-4}

Demostrar, usando la definición de límite, que
\(\lim_{x\to 1}\frac{3x+1}{2}=2\).

\end{exercise}

\begin{tcolorbox}[enhanced jigsaw, colbacktitle=quarto-callout-tip-color!10!white, breakable, toprule=.15mm, leftrule=.75mm, opacitybacktitle=0.6, left=2mm, colframe=quarto-callout-tip-color-frame, title=\textcolor{quarto-callout-tip-color}{\faLightbulb}\hspace{0.5em}{Solución}, colback=white, toptitle=1mm, bottomtitle=1mm, titlerule=0mm, coltitle=black, opacityback=0, arc=.35mm, rightrule=.15mm, bottomrule=.15mm]
Para cualquier \(\varepsilon>0\) existe
\(\delta=\frac{2}{3}\varepsilon\), tal que si
\(|x-1|<\delta=\frac{2}{3}\varepsilon\) se tiene

\[
\left|\frac{3x+1}{2}-2\right|= \left|\frac{3x+1-4}{2}\right| = \left|\frac{3x-3}{2}\right| = \left|\frac{3(x-1)}{2}\right| = \frac{3}{2}|x-1|< \frac{3}{2}\frac{2}{3}\varepsilon = \varepsilon.
\]
\end{tcolorbox}

\leavevmode\vadjust pre{\hypertarget{exr-5}{}}%
\begin{exercise}[]\label{exr-5}

Sabiendo que \(\lim_{x\to 0}(1+x)^{1/x}=e\), demostrar que las
siguientes funciones son infinitésimos equivalentes en \(x=0\):

\begin{enumerate}
\def\labelenumi{\alph{enumi}.}
\tightlist
\item
  \(\ln(1+x)\) y \(x\).
\item
  \(e^x-1\) y \(x\).
\end{enumerate}

\end{exercise}

\begin{tcolorbox}[enhanced jigsaw, colbacktitle=quarto-callout-tip-color!10!white, breakable, toprule=.15mm, leftrule=.75mm, opacitybacktitle=0.6, left=2mm, colframe=quarto-callout-tip-color-frame, title=\textcolor{quarto-callout-tip-color}{\faLightbulb}\hspace{0.5em}{Solución}, colback=white, toptitle=1mm, bottomtitle=1mm, titlerule=0mm, coltitle=black, opacityback=0, arc=.35mm, rightrule=.15mm, bottomrule=.15mm]
Para que dos funciones \(f\) y \(g\) sean infinitésimos equivalentes en
\(x=0\) se tiene que cumplir que \(\lim_{x\to 0}\frac{f(x)}{g(x)}=1\).

\begin{enumerate}
\def\labelenumi{\alph{enumi}.}
\item
  \begin{align*}
  \lim_{x\to 0}\frac{\ln(1+x)}{x} &= \lim_{x\to 0}\frac{1}{x}\ln(1+x) = \lim_{x\to 0}\ln\left((1+x)^{1/x}\right)\\  
  &= \ln\left(\lim_{x\to 0} (1+x)^{1/x}\right) = \ln(e) = 1.
  \end{align*}
\item
  Haciendo uso del resultado anterior se tiene
\end{enumerate}

\[\lim_{x\to 0}\frac{e^x-1}{x} = \lim_{x\to 0}\frac{e^{\ln(x+1)}-1}{x} = \lim_{x\to 0}\frac{x+1-1}{x} = \lim_{x\to 0} \frac{x}{x} = \lim_{x\to 0} 1 = 1.\]
\end{tcolorbox}

\leavevmode\vadjust pre{\hypertarget{exr-6}{}}%
\begin{exercise}[]\label{exr-6}

Determinar las asíntotas de la función
\(f(x)=\ln\left(\frac{1}{x}+1\right)x^2\).

\end{exercise}

\begin{tcolorbox}[enhanced jigsaw, colbacktitle=quarto-callout-tip-color!10!white, breakable, toprule=.15mm, leftrule=.75mm, opacitybacktitle=0.6, left=2mm, colframe=quarto-callout-tip-color-frame, title=\textcolor{quarto-callout-tip-color}{\faLightbulb}\hspace{0.5em}{Solución}, colback=white, toptitle=1mm, bottomtitle=1mm, titlerule=0mm, coltitle=black, opacityback=0, arc=.35mm, rightrule=.15mm, bottomrule=.15mm]
El dominio de la función es
\(\operatorname{Dom}(f) = \mathbb{R}-[-1,0]\) de modo que solo puede
haber asíntotas verticales a la izquierda de \(-1\) o a la derecha de
\(0\). Veamos primero, qué pasa con el límite por la izquierda en
\(-1\).

\[
\lim_{x\to -1^-}f(x) = \lim_{x\to -1^-}\ln\left(\frac{1}{x}+1\right)x^2 = \ln\left(\frac{1}{-1}+1\right)(-1)^2 = \ln(0) = -\infty.
\]

Por tanto, \(f\) tiene una asíntota vertical por la izquierda en
\(x=-1\).

Veamos ahora, qué pasa con el límite por la derecha en \(0\).

\begin{align*}
\lim_{x\to 0^+}f(x) &= \lim_{x\to 0^+}\ln\left(\frac{1}{x}+1\right)x^2 = \lim_{x\to 0^+}\ln\left(\frac{x+1}{x}\right)x^2 \\ 
&= \lim_{x\to 0^+} (\ln(x+1)-\ln(x))x^2 \\ 
&= \lim_{x\to 0^+}\ln(x+1)x^2 - \lim_{x\to 0^+}\ln(x)x^2\\ 
&= \ln(0+1)0^2 - \lim_{x\to 0^+}\frac{\ln(x)}{1/x^2} \\
&= 0 - \lim_{x\to 0^+}\frac{(\ln(x))'}{(1/x^2)'} = -\lim_{x\to 0^+} \frac{1/x}{-2/x^3} = \tag{L'Hôpital}\\
&= -\lim_{x\to 0^+} \frac{x^3}{-2x} = \lim_{x\to 0^+}\frac{x^2}{2} = 0.
\end{align*}

Por lo tanto, \(f\) no tiene asíntota vertical en \(x=0\).

Para ver si hay asíntotas horizontales estudiamos los límites en el
infinito.

\begin{align*}
\lim_{x\to \infty} f(x) &= \lim_{x\to \infty}\ln\left(\frac{1}{x}+1\right)x^2 = \lim_{x\to\infty} \frac{\ln(x^{-1}+1)}{x^{-2}}\\ 
&= \lim_{x\to\infty} \frac{(\ln(x^{-1}+1))'}{(x^{-2})'} \tag{L'Hôpital} = \lim_{x\to\infty} \frac{\frac{1}{x^{-1}+1}(-1)x^{-2}}{-2x^{-3}}\\  
&= \lim_{x\to \infty} \frac{x}{2(x^{-1}+1)} = \infty.
\end{align*}

Por tanto, \(f\) no tiene asíntota horizontal en \(\infty\). Veamos
ahora qué ocurre en \(-\infty\).

\begin{align*}
\lim_{x\to -\infty} f(x) &= \lim_{x\to -\infty}\ln\left(\frac{1}{x}+1\right)x^2 = \lim_{x\to -\infty} \frac{\ln(x^{-1}+1)}{x^{-2}}\\ 
&= \lim_{x\to -\infty} \frac{(\ln(x^{-1}+1))'}{(x^{-2})'} \tag{L'Hôpital} = \lim_{x\to -\infty} \frac{\frac{1}{x^{-1}+1}(-1)x^{-2}}{-2x^{-3}}\\ 
&= \lim_{x\to -\infty} \frac{x}{2(x^{-1}+1)} = -\infty.
\end{align*}

Luego, \(f\) tampoco tiene asíntota vertical en \(-\infty\).

Finalmente, veamos si \(f\) tiene asíntotas oblicuas.

\begin{align*}
\lim_{x\to \infty} \frac{f(x)}{x} &= \lim_{x\to \infty}\frac{\ln\left(\frac{1}{x}+1\right)x^2}{x} = \lim_{x\to \infty}\ln\left(\frac{1}{x}+1\right)x \\
&= \lim_{x\to \infty} \ln\left(\left(\frac{1}{x}+1\right)^x\right) = \ln\left(\lim_{x\to \infty} \left(\frac{1}{x}+1\right)^x\right)\\ 
&= \ln(e)=1
\end{align*}

Por tanto, \(f\) tiene asíntota vertical en \(\infty\) con pendiente
\(b=1\). Para obtener el término independiente de la asíntota,
calculamos el siguiente límite.

\begin{align*}
\lim_{x\to \infty} f(x)-x &= \lim_{x\to \infty}\ln\left(\frac{1}{x}+1\right)x^2-x\\  
&= \lim_{x\to \infty}(\ln(x^{-1}+1)x-1)x \\
&= \lim_{x\to \infty}\frac{\ln(x^{-1}+1)x-1}{x^{-1}}\\  
&=  \lim_{x\to \infty}\frac{(\ln(x^{-1}+1)x-1)'}{(x^{-1})'} \tag{L'Hôpital} \\
&= \lim_{x\to \infty}\frac{\frac{-1}{(x^{-1}+1)x^2}x+\ln(x^{-1}+1)}{(-1)x^{-2}}\\ 
&= \lim_{x\to \infty}\frac{\frac{-1}{(x+1)}+\ln(x^{-1}+1)}{-x^{-2}} \\
&= \lim_{x\to \infty}\frac{\left(\frac{-1}{(1+x)}+\ln(x^{-1}+1)\right)'}{(-x^{-2})'}\\ 
&= \lim_{x\to \infty}\frac{\frac{1}{(x+1)^2}-\frac{1}{(x^{-1}+1)x^2}}{2x^{-3}}\tag{L'Hôpital}\\ 
&= \lim_{x\to \infty}\frac{\frac{1}{(x+1)^2}-\frac{1}{(x+1)x}}{2x^{-3}} = \lim_{x\to \infty}\frac{\frac{-1}{(x+1)^2x}}{2x^{-3}} \\
&= \lim_{x\to \infty}\frac{-x^3}{2(x^3+2x^2+x)} = \frac{-1}{2}
\end{align*}

Así pues, \(f\) tiene una asíntota oblicua \(y=x-\frac{1}{2}\) en
\(\infty\).

Del mismo modo se prueba que esta misma recta también es asíntota
oblicua de \(f\) en \(-\infty\).
\end{tcolorbox}



\end{document}
